\documentclass[12pt]{amsart}   	% use "amsart" instead of "article" for AMSLaTeX format
\usepackage{geometry}                		% See geometry.pdf to learn the layout options. There are lots.
\geometry{a4paper}                   		% ... or a4paper or a5paper or ... 
%\geometry{landscape}                		% Activate for rotated page geometry
\usepackage[parfill]{parskip}    		% Activate to begin paragraphs with an empty line rather than an indent
\usepackage{graphicx}				% Use pdf, png, jpg, or eps§ with pdflatex; use eps in DVI mode
								% TeX will automatically convert eps --> pdf in pdflatex		
\usepackage{amssymb}
\usepackage{inputenc}
\usepackage{centernot}
%SetFonts
\usepackage{amsmath}
\newcommand{\me}{
    \author{Abhay Shankar K : cs21btech11001}
\maketitle
}

\begin{document}
    \title{Lab8 : Report}
    \me

    Program flow:
    \begin{itemize}
        \item Program reads from default input file (for ease of debugging) input.txt.
        \item numpy.loadtxt() converts the 0x-prefixed hexadecimal strings in the file
         to integers, storres them in a list s[].
        \item One pass through s generates another list ops, which contain only the opcodes of all instructions.
        \item Using a dict, these opcodes are matched to their respective handlers in a key-value pair.
        \item Each handler extracts the various fields of the instruction, and uses more dicts to determine the exact instruction being executed (based on funct3 and funct7).
        \item For the shift instructions, some references indicate another field named funct6, to differentiate berween arithmetic and logical shifts. This is not treated as a separate field and extracted later.
        \item For each line of machine code, the required handler returns a string containing the corresponding assembly. This gets stored in a list, temp.
        \item One pass through the assembly code allows us to look for branches and insert labels. The naming convention is L\{n\} for the n'th encountered branch/jump. The labelled code is stored in out[].
        \item The contents of out are printed to the terminal (for ease of debugging) as well as the default output file, output.txt.
    \end{itemize}

    Testing:
    \begin{itemize}
        \item The code was tested using three assembly programs provided for Lab7 : Caches. 
        \item Testing of signed correctness required some modifications to the given programs, and input.txt still contains those modifications.
    \end{itemize}
\end{document}